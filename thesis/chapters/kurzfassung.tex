Eine der großen Harausforderungen der Automobilindustrie ist die konstant zunehmende Komplexität von elektronischen Systemen und die dadurch entstehende Fehlerfortpflanzung aufgrund der weitgehenden Vernetzung dieser. Der Bereich der funktionalen Sicherheit beschäftigt sich mit der Vermeidung von nicht tolerierbaren Risiken im Fehlerfall. Dies wird durch die Identifizierung von möglichen Gefahren, der Abschätzung von potentiellen Risiken und, basierend darauf, der Entwicklung von notwendigen Gegenmaßnahmen gewährleistet. Diese Prozesse setzen ein detailiertes und umfassendes Verständnis der betrachteten elektronischen Systeme voraus.

Die Serviceorientierte Architektur ist ein Designprinzip, das sich auf das Konzept der Wiederverwendung von Software konzentriert. Dies wird durch die Implementierung von Funktionalitäten als technologieunabhängige und lose gekoppelte Services bewerkstelligt. Obwohl dieser Architekturtyp bereits weitgehend für Webanwendungen eingesetzt wird, hat er noch nicht in sicherheitskritische eingebettete Systeme Einzug gehalten.

Diese Bachelorarbeit untersucht die Verwendung von Serviceorientierten Architekturen in solchen Systemen und legt dabei einen Schwerpunkt auf funktionale Sicherheit und Fehlertoleranz. Der erste Teil der Arbeit definiert einige Begriffe, die für das Verständnis der vorgestellten Konzepte ausschlaggebend sind, und untersucht sie im Bezug auf eingebettet Systeme. Darauf basierend wird deffiniert was \emph{Sicherheit as Service} im Allgemeinen bedeuten kann, und wie die schlussendliche Implementierung in ein bestehendes System aussehen könnte, um die Auflagen des \mbox{ISO 26262} Standards zu erfüllen. Der \mbox{ISO 26262} Standard ist ein internationaler Standard für sicherheitskritische elektronische Systeme in Fahrzeugen mit einem maximal zulässigen Gesamtgewicht bis 3,500kg.