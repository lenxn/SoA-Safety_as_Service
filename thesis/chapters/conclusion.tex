In order to optimise the amount and maintainability of code, the concept of software reuse has been present since the very early days of software development itself. The SoA approach, which is treated within this thesis, is the state of the art concept in terms of software reuse. The underlying idea with this principle is that functionalities are implemented by means of so called services, which hide their internal logic and allow connections only through well-defined interfaces.

Although already widely applied in other domains, the SoA concept has not yet been applied for ESs like vehicles or aircraft yet. On the one hand, because of pure technical constraints due to the implemented hardware which is not optimised for the application of the SoA design paradigm. On the other hand, this is due to the safety-critical aspect of such systems. The \mbox{ISO 26262} standard does not issue any requirements concerning services or the like. Generally, functional safety in \textbf{safety-critical embedded systems} is an area with various unsolved questions and issues as it is stated in this thesis. A third drawback, which has not been mentioned so far, is the economic factor. The SoA for automotive requires dedicated hardware, which needs to be developed and adopted accordingly. In turn, this leads to high development and testing costs, before there is any benefit observable. At the same time it is hard to communicate the advantages of a vehicle with a SoA inside to the end user. The benefits become only obvious when there is a lot of environment and other vehicles which allow connection and communication with. In other words, all these concepts and ideas need a huge amount of resources and promotion to get them started in a useful way.

However, this architectural paradigm offers a wide range of advantages and benefits and is perhaps a necessary prerequisite for next generation's vehicles and aircraft. Hence it is no surprise that a lot of research has been conducted recently with this respect. The \mbox{Work Package 1} (\textbf{Embedded Systems Architectures}) of the EMC2 project is currently in the first year of a total of three. This indicates that an actual implementation of a SoA in a mass produced transportation system may still lie quite some time ahead. Nevertheless, the emerging of this kind of technology seems just a matter of time.