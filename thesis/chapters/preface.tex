This thesis was written as result of an internship at VIRTUAL VEHICLE at Inffeldgasse, Graz from April to July 2015, which was conducted as part of the Degree Programme in Aviation at FH Joanneum in Graz, Austria.

The VIRTUAL VEHICLE research center is an international company, specialized in automotive and rails industry. In total, it has more than 200 employees and concentrates on the four main research areas \emph{Thermo- \& Fluid Dynamics}, \emph{Mechanics \& Materials}, \emph{NVH \& Friction} and \emph{E/E \& Software}. For the time of my employment I worked as member of the Electrics/Electronics (E/E) \& Software area, with an emphasis on functional safety. I was assigned to the European project EMC2, a part of the ARTEMIS programme which focuses on embedded multi-core systems for mixed criticality applications in dynamic and changeable real-time environments. 

This thesis consists to a great extent of two documents, which were produced during this internship. The first document was a glossary, which aims at defining certain terms and unifying the opinions from different research areas. The second document contained an extensive investigation on how the service oriented architecture paradigm can be applied in a safety-critical embedded systems like vehicles.

Although the employment was oriented towards the automotive industry, the investigated functional safety and fault tolerance concepts reappear in a very similar way in other engineering disciplines, like aviation. Furthermore, the service oriented approach, which is here considered with respect to automotive, could also become an important issue for aviation in near future.

At this point I want to thank my supervisor from FH JOANNEUM, FH-Prof. Dipl.Ing. Dr. Holger Flühr, my supervisor provided by the company, Dipl.Ing. Helmut Martin, as well as my project team members Dipl.Ing.Dr. Andrea Leitner and Mr. Mario Driussi, who supported me in developing the ideas and concepts featured in this thesis during numerous meetings and discussions.