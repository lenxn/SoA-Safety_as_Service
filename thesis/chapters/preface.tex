This thesis was written as result of the internship at VIRTUAL VEHICLE at Inffeldgasse, Graz from April to July 2015, which was conducted as part of the Degree Programme in Aviation at FH Joanneum in Graz, Austria.

The VIRTUAL VEHICLE research center is an international company, specialized in automotive and rails industry. In total, it has more than 200 employees, splitted up in four main research areas. For the time of my employment, I worked as part of the Electrics/Electronics (E/E) \& Software Area, ... more specifically .. functional safety.

During my employment is was part of the EMC2 project, a European project, ....
my part soa
\\
\\
This thesis consists mainly of two documents, which were produced during my internship. The first one is a glossary, which aims at defining certain terms and unifying the opinions from different research areas. The second document, was an extensive investigation on how the service oriented architecture paradigm can be applied in a safety-critical embedded system. In detail, with a vehicle as the system.
\\
\\
Although the employment was oriented towards the automotive industry, the functional safety and fault tolerance concepts, reappear in a very similar way in other engineering disciplines, like aviation. Furthermore, the service oriented approach, which is considered in terms of automotive, will most likely become also an important issue for aviation in near future.

Furthermore, the location of the company at Inffeldgasse was another big advantage, for it allowed the conduction of various courses at the Technical University of Graz, alongside the employment.
\\
\\
At this point I want to thank my supervisor from FH JOANNEUM, FH-Prof. Dipl.Ing. Dr. Holger Flühr, as well as my supervisor provided by the company, Dipl.Ing. Helmut Martin. ... and also Dipl.Ing.Dr. Andrea Leitner and Mr. Mario Driussi, which helped develop the ideas and concepts featured in this thesis during numerous meetings and discussions.