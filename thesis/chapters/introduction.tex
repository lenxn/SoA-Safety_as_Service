\label{ch:introduction}
The SoA (Service oriented Architecture) design principle is a promising design philosophy, which features a lot of advantages over the static and vendor dependent architectures, which are implemented in today's vehicles and aircraft. Hence, it is not surprising that there have already been various projects dealing with the application of SoAs in real-time embedded systems. Those include \emph{SIRENA}, \emph{SOCRATES}, \emph{OASiS}, \emph{MORE}, \emph{RUNES} and \emph{$\varepsilon$SOA} \cite{scholz} \cite{sommer} \cite{buckl}. However, they do not address the necessary functional safety and fault tolerance requirements, which are preceding in vehicles. \textbf{Work Package 1}, ``Embedded System Architectures'', of the EMC2 project is dedicated to the investigation of these very issues, for this design paradigm might give way to a new generation of vehicles, which are able to interconnect and operate autonomously.

One of the most important sources for this thesis was the \mbox{ISO 26262} standard. Emerging from the \mbox{IEC 61508} standard, the \mbox{ISO 26262} standard is an international functional safety standard for series production passenger cars with a maximum gross weight of 3.500 kg. It is the state of the art for vehicles and does not set any requirements in terms of performance of equipment, but is only concerned with possible malfunctions. In total the standard features ten parts. Of particular relevance have been \textbf{part 1}, which defines all the related terms and vocabulary, \textbf{part 3} which deals with the concept phase, and \textbf{part 4}, which is dedicated  to the development at system level \cite{iso26262:1} \cite{iso26262:3} \cite{iso26262:4}.

The standard does not issue any regulations concerning SoA, but only provides certain requirements, which have to be fulfilled. Nevertheless, there are no prescriptions on how they should be fulfilled \cite{iso26262:course1}.

Another important source of information was \mbox{AUTOSAR}. The term \mbox{AUTOSAR} is ambiguous, for it can denote either the technical product (\textbf{AUT}omotive \textbf{O}pen \textbf{S}ystem \textbf{AR}chitecture), or the related development partnership \cite{autosar_rs_main}. The partnership was founded in 2003, with its members covering more than 80\% of the production of cars worldwide \cite{kirschke_biller2011} \cite{schmerler2012}. They provide a standard, which aims at establishing an industry norm for automotive software architecture. This allows different partners, as well as suppliers and manufacturers, to collaborate without any obstacles in terms of languages or methodologies. In detail, this is achieved by the definition of a unified \emph{software architecture} and \emph{software development methodology}, as well as \emph{standardised application interfaces}. By stressing the decoupling of hardware and software, the standard gives way to software reuse on different hardware platforms \cite{kirschke_biller2011} \cite{schmerler2012}, what is in compliance with the SoA design principles. Thus, many of the terms, treated within this thesis are influenced by the viewpoint of \mbox{AUTOSAR}.
