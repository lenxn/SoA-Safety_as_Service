Due to the very different comprehension of certain terms by different people from different fields of research 
The first part of my work during the internship was the creation of a glossary describing certain given terms
... in order to unify the understanding of these terms and create a basis for discussions ans researches.

- The glossary was later taken as standard for this tasks ...
- and deals with the terms: \emph{service}, \emph{system}, \emph{system of systems}, \emph{architecture}, \emph{service-oriented architecture}, \emph{configuration}, \emph{static reconfiguration}, \emph{dynamic reconfiguration}, \emph{inter-core communication}, \emph{intra-core communication} and \emph{binding}.
The most important parts of this glossary will be covered within this chapter.
- furthermore a database with the used literature was created in order to provide the necessary references for the glossary and provide a ... where to find further information for the employees. 

- disambiguity safety and security
- what is safety and fault tolerance

\section{Related instances}
\subsection{ISO 262662 international standard}
\subsection{MISRA}
\subsection{AUTOSAR}
\section{service}

\section{System}

\section{Component}

\section{Service}

\section{Architecture}

\section{Service oriented Architecture}
\subsection{Definition}
\subsection{Historic Development}
\subsection{Structure of SoA}
\subsection{Service Composition}

\section{Communication in SoA}

\section{Dependability}
\subsection{Reliability}
\subsection{Availability}

\section{Functional Safety}
\subsection{Safety related terminology}
\subsection{Definition of safety}
\subsection{Fault tolerance}