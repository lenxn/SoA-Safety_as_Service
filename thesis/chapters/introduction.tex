Due to the very different comprehension of certain terms by different people from different fields of research 
The first part of my work during the internship was the creation of a glossary describing certain given terms
... in order to unify the understanding of these terms and create a basis for discussions ans researches.

- The glossary was later taken as standard for this tasks ...
- and deals with the terms: \emph{service}, \emph{system}, \emph{system of systems}, \emph{architecture}, \emph{service-oriented architecture}, \emph{configuration}, \emph{static reconfiguration}, \emph{dynamic reconfiguration}, \emph{inter-core communication}, \emph{intra-core communication} and \emph{binding}.
The most important parts of this glossary will be covered within this chapter.
- furthermore a database with the used literature was created in order to provide the necessary references for the glossary and provide a ... where to find further information for the employees. 

- disambiguity safety and security
- what is safety and fault tolerance

\section{service}

\section{bindings}

\section{literature database}
- The literature database was written with SqL
- contains most of the used references
- which are of relevance
- helps to gain further knowledge on the various topics

- sqldump in the appendix
- description of the literature and estimation of their relevance
- rating of the ressources (1-5)


%%%%%%%%%%%%%%%%%%%%%%%%%%
