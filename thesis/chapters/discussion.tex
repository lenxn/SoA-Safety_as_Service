For the future of automotive with self-driving interconnected cars, the application of a service oriented architecture paradigm will be the logic choice. However, there are various obstacles which delay or even prevent this development.

Those are the technical constraints due to the implemented hardware, which is not optimised for the application of the SoA design paradigm, as well as the safety-critical aspect of vehicles. The \mbox{ISO 26262} standard does not issue any requirements concerning services or the like. Generally, functional safety in \textbf{safety-critical embedded systems} is an area with various unsolved questions and issues as it is stated in this thesis.

A third drawback, which has not been mentioned so far, is the economic factor. The SoA for automotive requires dedicated hardware, which needs to be developed and adopted accordingly. In turn, this leads to high development and testing costs, before there is any benefit observable. At the same time it is hard to communicate the advantages of a vehicle with a SoA inside to the end user. The benefits become only obvious, when there is a lot of environment and other vehicles which allow connection and communication with. In other words, all these concepts and ideas need a huge amount of resources and promotion to get them started in a useful way.