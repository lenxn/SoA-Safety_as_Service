One of the major issues in the automotive industry is the constant growing complexity of E/E (Electrics/Electronics) systems and the thereof resulting fault propagation due to the strong interconnection of the systems. The area of \emph{functional safety} is concerned with the prevention of non tolerable risks in event of error. This is conducted by identifying possible hazards, estimating the potential risks and developing necessary countermeasures, based on these investigations. The requirement therefore is an accurate and thorough comprehension of the observed E/E system.

The service oriented architecture is a design paradigm, which features the concept of software reuse by implementing functionalities as technology independent and loosely coupled services. Although the design paradigm is already standard for Web applications, it has not yet been applied in safety-critical embedded systems.

Thus, this Bachelor's thesis investigates the applicability of service oriented architectures for such systems, with a focus on \emph{functional safety} and \emph{fault tolerance} context. 

The first part of the thesis is mainly a glossy, which defines certain terms, which are critical for the subsequent presented concepts. On this basis it is defined what \emph{safety as a service} can mean in general, and how the actual implementation in a given safety-critical system may look like, in order to meant with the requirements specified in the ISO 26262 standard. This is the safety standard for safety-critical E/E systems in vehicles with a maximum gross weight of 3500 kg.